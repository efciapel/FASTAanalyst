\documentclass[11pt,a4paper]{mwart}
\usepackage[T1]{fontenc}
\usepackage[utf8]{inputenc}
\usepackage{polski}
\setlength{\parindent}{1cm}
\def\labelitemi{$\bullet$}
\linespread{1.4}
\usepackage[a4paper, left= 2.5cm, right=2.5cm, top=3.5cm, bottom=3.5cm, headsep=1.2cm]{geometry}
\usepackage{graphicx}
\usepackage{tikz}
\usepackage{xcolor}
\definecolor{font-color}{RGB}{224, 226, 228}
\definecolor{bg-color}{RGB}{41, 49, 52}
\definecolor{string-color}{RGB}{235, 118, 0}
\usepackage{listings}

\begin{document}
\lstset{ %
  backgroundcolor=\color{bg-color},   % choose the background color; you must add \usepackage{color} or \usepackage{xcolor}
  basicstyle=\sffamily\bfseries\small\color{font-color},        % the size of the fonts that are used for the code
  breakatwhitespace=true,         % sets if automatic breaks should only happen at whitespace
  breaklines=true,                 % sets automatic line breaking
  captionpos=b,                    % sets the caption-position to bottom
  commentstyle=\color{green},    % comment style
  deletekeywords={...},            % if you want to delete keywords from the given language
  escapeinside={\%*}{*)},          % if you want to add LaTeX within your code
  extendedchars=true,              % lets you use non-ASCII characters; for 8-bits encodings only, does not work with UTF-8
  %frame=single,	                   % adds a frame around the code
  %keepspaces=true,                 % keeps spaces in text, useful for keeping indentation of code (possibly needs columns=flexible)
  keywordstyle=\color{blue},       % keyword style
  language=Python,                 % the language of the code
  otherkeywords={*,...},           % if you want to add more keywords to the set
  numbers=left,                    % where to put the line-numbers; possible values are (none, left, right)
  numbersep=5pt,                   % how far the line-numbers are from the code
  numberstyle=\tiny\color{gray}, % the style that is used for the line-numbers
  %rulecolor=\color{blue},         % if not set, the frame-color may be changed on line-breaks within not-black text (e.g. comments (green here))
  showspaces=false,                % show spaces everywhere adding particular underscores; it overrides 'showstringspaces'
  showstringspaces=false,          % underline spaces within strings only
  showtabs=false,                  % show tabs within strings adding particular underscores
  stepnumber=1,                    % the step between two line-numbers. If it's 1, each line will be numbered
  stringstyle=\color{string-color},     % string literal style
  tabsize=2,	                   % sets default tabsize to 2 spaces
  title=\lstname                   % show the filename of files included with \lstinputlisting; also try caption instead of title
}

	\begin{center}
		\begin{huge}
			\textbf{FASTAanalyst}\\
		\end{huge}
		\begin{Large}
			Biblioteka do analizy sekwencji DNA z plików FASTA\\
		\end{Large}
		\begin{large}
			\textsc{Specyfikacja}\\
		\end{large}
		Październik 2016\\
		\textit{M. Kępska, E. Król, A. Osina}
		\vspace{.5cm}
		\rule{\textwidth}{1pt}\\\rule[.5cm]{\textwidth}{1pt}
	\end{center}
	\section{Wprowadzenie}
		Biblioteka FASTAanalyst jest przydatnym narzędziem bioinformatycznym.
		Służy do odczytywania informacji z plików .FASTA i poddawania ich 
		odpowiednim analizom. Biblioteka dostarcza takich analiz jak obliczanie
		temperatury topnienia, wyliczanie masy cząsteczkowej czy wyliczanie 
		punktów przyrównania dwóch sekwencji.
	\section{Funkcjonalności}
		\subsection{read\_FASTA}
			Wczytuje pliki .FASTA i zwraca czystą sekwencję DNA (bez informacji
			z nagłówka). Jeżeli użytkownik użyje tej funkcji do wczytania surowego
			DNA, argumenty zostaną automatycznie przekazane funkcji 
			\textsf{read\_DNA} (patrz niżej).
		\subsection{read\_DNA}
			Wczytuje surową sekwencję DNA. Funkcja ta ma być pomocna w przypadku 
			wczytywaniu krótkich sekwencji DNA. Jeżeli użytkownik użyje tej funkcji 
			do wczytania pliku .FASTA, argumenty zostaną automatycznie przekazane 	
			funkcji \textsf{read\_FASTA}.
		\subsection{lenght}
			Zwraca długość podanej sekwencji.
		\subsection{nucleotides\_occurence}
			Podaje procentowe występowanie danego nukleotydu w sekwencji lub grupy
			nukleotydów. Funkcja może przyjąć wiele argumentów, więc użytkownik
			uzyskuje wiele informacji za jednym razem.\\
			\indent Przykładowo: użytkownik chce dowiedzieć się jaki udział
			sekwencji mają osobno nukleotydy A, T, C, G oraz grupy nukleotydów AT 
			oraz CG. Podaje odpowiednie argumenty:
			\begin{lstlisting}
['a', 't', 'c', 'g', 'at', 'cg']
			\end{lstlisting}
			a funkcja zwraca mu słownik:
			\begin{lstlisting}
{'A': 32.35, 'T': 23.53, 'C': 26.47, 'G': 17.65, 'AT': 55.88,
'GC': 44.12}
			\end{lstlisting}
		\subsection{complement}
			Zwraca nić komplementarną do podanej sekwencji
		\subsection{reverse\_complement}
			Zwraca odwróconą nić komplementarną.
		\subsection{molecular\_mass}
			Zwraca masę cząsteczkową łańcucha.
		\subsection{melting\_point}
			Zwraca temperaturę potrzebną do denaturacji podanego łańcucha DNA.
			Funkcja korzysta ze wzoru: $4(G+C)+2(A+T)$, gdzie $G$, $C$, $A$ i $T$
			są, odpowiednio, liczbą zasad guaninowych, cytozynowych, adeninowych i 
			tyminowych (lub uracylowych, w przypadku łańcuchów RNA).
		\subsection{alignment}
			Zwraca score przyrównania dwóch sekwencji. Wykorzystuje algorytm
			Needlemana-Wunscha.
\end{document}